
% Default to the notebook output style

    


% Inherit from the specified cell style.




    
\documentclass[11pt]{article}

    
    
    \usepackage[T1]{fontenc}
    % Nicer default font (+ math font) than Computer Modern for most use cases
    \usepackage{mathpazo}

    % Basic figure setup, for now with no caption control since it's done
    % automatically by Pandoc (which extracts ![](path) syntax from Markdown).
    \usepackage{graphicx}
    % We will generate all images so they have a width \maxwidth. This means
    % that they will get their normal width if they fit onto the page, but
    % are scaled down if they would overflow the margins.
    \makeatletter
    \def\maxwidth{\ifdim\Gin@nat@width>\linewidth\linewidth
    \else\Gin@nat@width\fi}
    \makeatother
    \let\Oldincludegraphics\includegraphics
    % Set max figure width to be 80% of text width, for now hardcoded.
    \renewcommand{\includegraphics}[1]{\Oldincludegraphics[width=.8\maxwidth]{#1}}
    % Ensure that by default, figures have no caption (until we provide a
    % proper Figure object with a Caption API and a way to capture that
    % in the conversion process - todo).
    \usepackage{caption}
    \DeclareCaptionLabelFormat{nolabel}{}
    \captionsetup{labelformat=nolabel}

    \usepackage{adjustbox} % Used to constrain images to a maximum size 
    \usepackage{xcolor} % Allow colors to be defined
    \usepackage{enumerate} % Needed for markdown enumerations to work
    \usepackage{geometry} % Used to adjust the document margins
    \usepackage{amsmath} % Equations
    \usepackage{amssymb} % Equations
    \usepackage{textcomp} % defines textquotesingle
    % Hack from http://tex.stackexchange.com/a/47451/13684:
    \AtBeginDocument{%
        \def\PYZsq{\textquotesingle}% Upright quotes in Pygmentized code
    }
    \usepackage{upquote} % Upright quotes for verbatim code
    \usepackage{eurosym} % defines \euro
    \usepackage[mathletters]{ucs} % Extended unicode (utf-8) support
    \usepackage[utf8x]{inputenc} % Allow utf-8 characters in the tex document
    \usepackage{fancyvrb} % verbatim replacement that allows latex
    \usepackage{grffile} % extends the file name processing of package graphics 
                         % to support a larger range 
    % The hyperref package gives us a pdf with properly built
    % internal navigation ('pdf bookmarks' for the table of contents,
    % internal cross-reference links, web links for URLs, etc.)
    \usepackage{hyperref}
    \usepackage{longtable} % longtable support required by pandoc >1.10
    \usepackage{booktabs}  % table support for pandoc > 1.12.2
    \usepackage[inline]{enumitem} % IRkernel/repr support (it uses the enumerate* environment)
    \usepackage[normalem]{ulem} % ulem is needed to support strikethroughs (\sout)
                                % normalem makes italics be italics, not underlines
    

    
    
    % Colors for the hyperref package
    \definecolor{urlcolor}{rgb}{0,.145,.698}
    \definecolor{linkcolor}{rgb}{.71,0.21,0.01}
    \definecolor{citecolor}{rgb}{.12,.54,.11}

    % ANSI colors
    \definecolor{ansi-black}{HTML}{3E424D}
    \definecolor{ansi-black-intense}{HTML}{282C36}
    \definecolor{ansi-red}{HTML}{E75C58}
    \definecolor{ansi-red-intense}{HTML}{B22B31}
    \definecolor{ansi-green}{HTML}{00A250}
    \definecolor{ansi-green-intense}{HTML}{007427}
    \definecolor{ansi-yellow}{HTML}{DDB62B}
    \definecolor{ansi-yellow-intense}{HTML}{B27D12}
    \definecolor{ansi-blue}{HTML}{208FFB}
    \definecolor{ansi-blue-intense}{HTML}{0065CA}
    \definecolor{ansi-magenta}{HTML}{D160C4}
    \definecolor{ansi-magenta-intense}{HTML}{A03196}
    \definecolor{ansi-cyan}{HTML}{60C6C8}
    \definecolor{ansi-cyan-intense}{HTML}{258F8F}
    \definecolor{ansi-white}{HTML}{C5C1B4}
    \definecolor{ansi-white-intense}{HTML}{A1A6B2}

    % commands and environments needed by pandoc snippets
    % extracted from the output of `pandoc -s`
    \providecommand{\tightlist}{%
      \setlength{\itemsep}{0pt}\setlength{\parskip}{0pt}}
    \DefineVerbatimEnvironment{Highlighting}{Verbatim}{commandchars=\\\{\}}
    % Add ',fontsize=\small' for more characters per line
    \newenvironment{Shaded}{}{}
    \newcommand{\KeywordTok}[1]{\textcolor[rgb]{0.00,0.44,0.13}{\textbf{{#1}}}}
    \newcommand{\DataTypeTok}[1]{\textcolor[rgb]{0.56,0.13,0.00}{{#1}}}
    \newcommand{\DecValTok}[1]{\textcolor[rgb]{0.25,0.63,0.44}{{#1}}}
    \newcommand{\BaseNTok}[1]{\textcolor[rgb]{0.25,0.63,0.44}{{#1}}}
    \newcommand{\FloatTok}[1]{\textcolor[rgb]{0.25,0.63,0.44}{{#1}}}
    \newcommand{\CharTok}[1]{\textcolor[rgb]{0.25,0.44,0.63}{{#1}}}
    \newcommand{\StringTok}[1]{\textcolor[rgb]{0.25,0.44,0.63}{{#1}}}
    \newcommand{\CommentTok}[1]{\textcolor[rgb]{0.38,0.63,0.69}{\textit{{#1}}}}
    \newcommand{\OtherTok}[1]{\textcolor[rgb]{0.00,0.44,0.13}{{#1}}}
    \newcommand{\AlertTok}[1]{\textcolor[rgb]{1.00,0.00,0.00}{\textbf{{#1}}}}
    \newcommand{\FunctionTok}[1]{\textcolor[rgb]{0.02,0.16,0.49}{{#1}}}
    \newcommand{\RegionMarkerTok}[1]{{#1}}
    \newcommand{\ErrorTok}[1]{\textcolor[rgb]{1.00,0.00,0.00}{\textbf{{#1}}}}
    \newcommand{\NormalTok}[1]{{#1}}
    
    % Additional commands for more recent versions of Pandoc
    \newcommand{\ConstantTok}[1]{\textcolor[rgb]{0.53,0.00,0.00}{{#1}}}
    \newcommand{\SpecialCharTok}[1]{\textcolor[rgb]{0.25,0.44,0.63}{{#1}}}
    \newcommand{\VerbatimStringTok}[1]{\textcolor[rgb]{0.25,0.44,0.63}{{#1}}}
    \newcommand{\SpecialStringTok}[1]{\textcolor[rgb]{0.73,0.40,0.53}{{#1}}}
    \newcommand{\ImportTok}[1]{{#1}}
    \newcommand{\DocumentationTok}[1]{\textcolor[rgb]{0.73,0.13,0.13}{\textit{{#1}}}}
    \newcommand{\AnnotationTok}[1]{\textcolor[rgb]{0.38,0.63,0.69}{\textbf{\textit{{#1}}}}}
    \newcommand{\CommentVarTok}[1]{\textcolor[rgb]{0.38,0.63,0.69}{\textbf{\textit{{#1}}}}}
    \newcommand{\VariableTok}[1]{\textcolor[rgb]{0.10,0.09,0.49}{{#1}}}
    \newcommand{\ControlFlowTok}[1]{\textcolor[rgb]{0.00,0.44,0.13}{\textbf{{#1}}}}
    \newcommand{\OperatorTok}[1]{\textcolor[rgb]{0.40,0.40,0.40}{{#1}}}
    \newcommand{\BuiltInTok}[1]{{#1}}
    \newcommand{\ExtensionTok}[1]{{#1}}
    \newcommand{\PreprocessorTok}[1]{\textcolor[rgb]{0.74,0.48,0.00}{{#1}}}
    \newcommand{\AttributeTok}[1]{\textcolor[rgb]{0.49,0.56,0.16}{{#1}}}
    \newcommand{\InformationTok}[1]{\textcolor[rgb]{0.38,0.63,0.69}{\textbf{\textit{{#1}}}}}
    \newcommand{\WarningTok}[1]{\textcolor[rgb]{0.38,0.63,0.69}{\textbf{\textit{{#1}}}}}
    
    
    % Define a nice break command that doesn't care if a line doesn't already
    % exist.
    \def\br{\hspace*{\fill} \\* }
    % Math Jax compatability definitions
    \def\gt{>}
    \def\lt{<}
    % Document parameters
    \title{notebook}
    
    
    

    % Pygments definitions
    
\makeatletter
\def\PY@reset{\let\PY@it=\relax \let\PY@bf=\relax%
    \let\PY@ul=\relax \let\PY@tc=\relax%
    \let\PY@bc=\relax \let\PY@ff=\relax}
\def\PY@tok#1{\csname PY@tok@#1\endcsname}
\def\PY@toks#1+{\ifx\relax#1\empty\else%
    \PY@tok{#1}\expandafter\PY@toks\fi}
\def\PY@do#1{\PY@bc{\PY@tc{\PY@ul{%
    \PY@it{\PY@bf{\PY@ff{#1}}}}}}}
\def\PY#1#2{\PY@reset\PY@toks#1+\relax+\PY@do{#2}}

\expandafter\def\csname PY@tok@w\endcsname{\def\PY@tc##1{\textcolor[rgb]{0.73,0.73,0.73}{##1}}}
\expandafter\def\csname PY@tok@c\endcsname{\let\PY@it=\textit\def\PY@tc##1{\textcolor[rgb]{0.25,0.50,0.50}{##1}}}
\expandafter\def\csname PY@tok@cp\endcsname{\def\PY@tc##1{\textcolor[rgb]{0.74,0.48,0.00}{##1}}}
\expandafter\def\csname PY@tok@k\endcsname{\let\PY@bf=\textbf\def\PY@tc##1{\textcolor[rgb]{0.00,0.50,0.00}{##1}}}
\expandafter\def\csname PY@tok@kp\endcsname{\def\PY@tc##1{\textcolor[rgb]{0.00,0.50,0.00}{##1}}}
\expandafter\def\csname PY@tok@kt\endcsname{\def\PY@tc##1{\textcolor[rgb]{0.69,0.00,0.25}{##1}}}
\expandafter\def\csname PY@tok@o\endcsname{\def\PY@tc##1{\textcolor[rgb]{0.40,0.40,0.40}{##1}}}
\expandafter\def\csname PY@tok@ow\endcsname{\let\PY@bf=\textbf\def\PY@tc##1{\textcolor[rgb]{0.67,0.13,1.00}{##1}}}
\expandafter\def\csname PY@tok@nb\endcsname{\def\PY@tc##1{\textcolor[rgb]{0.00,0.50,0.00}{##1}}}
\expandafter\def\csname PY@tok@nf\endcsname{\def\PY@tc##1{\textcolor[rgb]{0.00,0.00,1.00}{##1}}}
\expandafter\def\csname PY@tok@nc\endcsname{\let\PY@bf=\textbf\def\PY@tc##1{\textcolor[rgb]{0.00,0.00,1.00}{##1}}}
\expandafter\def\csname PY@tok@nn\endcsname{\let\PY@bf=\textbf\def\PY@tc##1{\textcolor[rgb]{0.00,0.00,1.00}{##1}}}
\expandafter\def\csname PY@tok@ne\endcsname{\let\PY@bf=\textbf\def\PY@tc##1{\textcolor[rgb]{0.82,0.25,0.23}{##1}}}
\expandafter\def\csname PY@tok@nv\endcsname{\def\PY@tc##1{\textcolor[rgb]{0.10,0.09,0.49}{##1}}}
\expandafter\def\csname PY@tok@no\endcsname{\def\PY@tc##1{\textcolor[rgb]{0.53,0.00,0.00}{##1}}}
\expandafter\def\csname PY@tok@nl\endcsname{\def\PY@tc##1{\textcolor[rgb]{0.63,0.63,0.00}{##1}}}
\expandafter\def\csname PY@tok@ni\endcsname{\let\PY@bf=\textbf\def\PY@tc##1{\textcolor[rgb]{0.60,0.60,0.60}{##1}}}
\expandafter\def\csname PY@tok@na\endcsname{\def\PY@tc##1{\textcolor[rgb]{0.49,0.56,0.16}{##1}}}
\expandafter\def\csname PY@tok@nt\endcsname{\let\PY@bf=\textbf\def\PY@tc##1{\textcolor[rgb]{0.00,0.50,0.00}{##1}}}
\expandafter\def\csname PY@tok@nd\endcsname{\def\PY@tc##1{\textcolor[rgb]{0.67,0.13,1.00}{##1}}}
\expandafter\def\csname PY@tok@s\endcsname{\def\PY@tc##1{\textcolor[rgb]{0.73,0.13,0.13}{##1}}}
\expandafter\def\csname PY@tok@sd\endcsname{\let\PY@it=\textit\def\PY@tc##1{\textcolor[rgb]{0.73,0.13,0.13}{##1}}}
\expandafter\def\csname PY@tok@si\endcsname{\let\PY@bf=\textbf\def\PY@tc##1{\textcolor[rgb]{0.73,0.40,0.53}{##1}}}
\expandafter\def\csname PY@tok@se\endcsname{\let\PY@bf=\textbf\def\PY@tc##1{\textcolor[rgb]{0.73,0.40,0.13}{##1}}}
\expandafter\def\csname PY@tok@sr\endcsname{\def\PY@tc##1{\textcolor[rgb]{0.73,0.40,0.53}{##1}}}
\expandafter\def\csname PY@tok@ss\endcsname{\def\PY@tc##1{\textcolor[rgb]{0.10,0.09,0.49}{##1}}}
\expandafter\def\csname PY@tok@sx\endcsname{\def\PY@tc##1{\textcolor[rgb]{0.00,0.50,0.00}{##1}}}
\expandafter\def\csname PY@tok@m\endcsname{\def\PY@tc##1{\textcolor[rgb]{0.40,0.40,0.40}{##1}}}
\expandafter\def\csname PY@tok@gh\endcsname{\let\PY@bf=\textbf\def\PY@tc##1{\textcolor[rgb]{0.00,0.00,0.50}{##1}}}
\expandafter\def\csname PY@tok@gu\endcsname{\let\PY@bf=\textbf\def\PY@tc##1{\textcolor[rgb]{0.50,0.00,0.50}{##1}}}
\expandafter\def\csname PY@tok@gd\endcsname{\def\PY@tc##1{\textcolor[rgb]{0.63,0.00,0.00}{##1}}}
\expandafter\def\csname PY@tok@gi\endcsname{\def\PY@tc##1{\textcolor[rgb]{0.00,0.63,0.00}{##1}}}
\expandafter\def\csname PY@tok@gr\endcsname{\def\PY@tc##1{\textcolor[rgb]{1.00,0.00,0.00}{##1}}}
\expandafter\def\csname PY@tok@ge\endcsname{\let\PY@it=\textit}
\expandafter\def\csname PY@tok@gs\endcsname{\let\PY@bf=\textbf}
\expandafter\def\csname PY@tok@gp\endcsname{\let\PY@bf=\textbf\def\PY@tc##1{\textcolor[rgb]{0.00,0.00,0.50}{##1}}}
\expandafter\def\csname PY@tok@go\endcsname{\def\PY@tc##1{\textcolor[rgb]{0.53,0.53,0.53}{##1}}}
\expandafter\def\csname PY@tok@gt\endcsname{\def\PY@tc##1{\textcolor[rgb]{0.00,0.27,0.87}{##1}}}
\expandafter\def\csname PY@tok@err\endcsname{\def\PY@bc##1{\setlength{\fboxsep}{0pt}\fcolorbox[rgb]{1.00,0.00,0.00}{1,1,1}{\strut ##1}}}
\expandafter\def\csname PY@tok@kc\endcsname{\let\PY@bf=\textbf\def\PY@tc##1{\textcolor[rgb]{0.00,0.50,0.00}{##1}}}
\expandafter\def\csname PY@tok@kd\endcsname{\let\PY@bf=\textbf\def\PY@tc##1{\textcolor[rgb]{0.00,0.50,0.00}{##1}}}
\expandafter\def\csname PY@tok@kn\endcsname{\let\PY@bf=\textbf\def\PY@tc##1{\textcolor[rgb]{0.00,0.50,0.00}{##1}}}
\expandafter\def\csname PY@tok@kr\endcsname{\let\PY@bf=\textbf\def\PY@tc##1{\textcolor[rgb]{0.00,0.50,0.00}{##1}}}
\expandafter\def\csname PY@tok@bp\endcsname{\def\PY@tc##1{\textcolor[rgb]{0.00,0.50,0.00}{##1}}}
\expandafter\def\csname PY@tok@fm\endcsname{\def\PY@tc##1{\textcolor[rgb]{0.00,0.00,1.00}{##1}}}
\expandafter\def\csname PY@tok@vc\endcsname{\def\PY@tc##1{\textcolor[rgb]{0.10,0.09,0.49}{##1}}}
\expandafter\def\csname PY@tok@vg\endcsname{\def\PY@tc##1{\textcolor[rgb]{0.10,0.09,0.49}{##1}}}
\expandafter\def\csname PY@tok@vi\endcsname{\def\PY@tc##1{\textcolor[rgb]{0.10,0.09,0.49}{##1}}}
\expandafter\def\csname PY@tok@vm\endcsname{\def\PY@tc##1{\textcolor[rgb]{0.10,0.09,0.49}{##1}}}
\expandafter\def\csname PY@tok@sa\endcsname{\def\PY@tc##1{\textcolor[rgb]{0.73,0.13,0.13}{##1}}}
\expandafter\def\csname PY@tok@sb\endcsname{\def\PY@tc##1{\textcolor[rgb]{0.73,0.13,0.13}{##1}}}
\expandafter\def\csname PY@tok@sc\endcsname{\def\PY@tc##1{\textcolor[rgb]{0.73,0.13,0.13}{##1}}}
\expandafter\def\csname PY@tok@dl\endcsname{\def\PY@tc##1{\textcolor[rgb]{0.73,0.13,0.13}{##1}}}
\expandafter\def\csname PY@tok@s2\endcsname{\def\PY@tc##1{\textcolor[rgb]{0.73,0.13,0.13}{##1}}}
\expandafter\def\csname PY@tok@sh\endcsname{\def\PY@tc##1{\textcolor[rgb]{0.73,0.13,0.13}{##1}}}
\expandafter\def\csname PY@tok@s1\endcsname{\def\PY@tc##1{\textcolor[rgb]{0.73,0.13,0.13}{##1}}}
\expandafter\def\csname PY@tok@mb\endcsname{\def\PY@tc##1{\textcolor[rgb]{0.40,0.40,0.40}{##1}}}
\expandafter\def\csname PY@tok@mf\endcsname{\def\PY@tc##1{\textcolor[rgb]{0.40,0.40,0.40}{##1}}}
\expandafter\def\csname PY@tok@mh\endcsname{\def\PY@tc##1{\textcolor[rgb]{0.40,0.40,0.40}{##1}}}
\expandafter\def\csname PY@tok@mi\endcsname{\def\PY@tc##1{\textcolor[rgb]{0.40,0.40,0.40}{##1}}}
\expandafter\def\csname PY@tok@il\endcsname{\def\PY@tc##1{\textcolor[rgb]{0.40,0.40,0.40}{##1}}}
\expandafter\def\csname PY@tok@mo\endcsname{\def\PY@tc##1{\textcolor[rgb]{0.40,0.40,0.40}{##1}}}
\expandafter\def\csname PY@tok@ch\endcsname{\let\PY@it=\textit\def\PY@tc##1{\textcolor[rgb]{0.25,0.50,0.50}{##1}}}
\expandafter\def\csname PY@tok@cm\endcsname{\let\PY@it=\textit\def\PY@tc##1{\textcolor[rgb]{0.25,0.50,0.50}{##1}}}
\expandafter\def\csname PY@tok@cpf\endcsname{\let\PY@it=\textit\def\PY@tc##1{\textcolor[rgb]{0.25,0.50,0.50}{##1}}}
\expandafter\def\csname PY@tok@c1\endcsname{\let\PY@it=\textit\def\PY@tc##1{\textcolor[rgb]{0.25,0.50,0.50}{##1}}}
\expandafter\def\csname PY@tok@cs\endcsname{\let\PY@it=\textit\def\PY@tc##1{\textcolor[rgb]{0.25,0.50,0.50}{##1}}}

\def\PYZbs{\char`\\}
\def\PYZus{\char`\_}
\def\PYZob{\char`\{}
\def\PYZcb{\char`\}}
\def\PYZca{\char`\^}
\def\PYZam{\char`\&}
\def\PYZlt{\char`\<}
\def\PYZgt{\char`\>}
\def\PYZsh{\char`\#}
\def\PYZpc{\char`\%}
\def\PYZdl{\char`\$}
\def\PYZhy{\char`\-}
\def\PYZsq{\char`\'}
\def\PYZdq{\char`\"}
\def\PYZti{\char`\~}
% for compatibility with earlier versions
\def\PYZat{@}
\def\PYZlb{[}
\def\PYZrb{]}
\makeatother


    % Exact colors from NB
    \definecolor{incolor}{rgb}{0.0, 0.0, 0.5}
    \definecolor{outcolor}{rgb}{0.545, 0.0, 0.0}



    
    % Prevent overflowing lines due to hard-to-break entities
    \sloppy 
    % Setup hyperref package
    \hypersetup{
      breaklinks=true,  % so long urls are correctly broken across lines
      colorlinks=true,
      urlcolor=urlcolor,
      linkcolor=linkcolor,
      citecolor=citecolor,
      }
    % Slightly bigger margins than the latex defaults
    
    \geometry{verbose,tmargin=1in,bmargin=1in,lmargin=1in,rmargin=1in}
    
    

    \begin{document}
    
    
    \maketitle
    
    

    
    \hypertarget{breath-alcohol-tests-in-ames-iowa-usa}{%
\subsection{1. Breath alcohol tests in Ames, Iowa,
USA}\label{breath-alcohol-tests-in-ames-iowa-usa}}

Ames, Iowa, USA is the home of Iowa State University, a land grant
university with over 36,000 students. By comparison, the city of Ames,
Iowa, itself only has about 65,000 residents. As with any other college
town, Ames has had its fair share of alcohol-related incidents. (For
example, Google `VEISHEA riots 2014'.) We will take a look at some
breath alcohol test data from Ames that is published by the State of
Iowa.

The data file `breath\_alcohol\_ames.csv' contains 1,556 readings from
breath alcohol tests administered by the Ames and Iowa State University
Police Departments from January 2013 to December 2017. The columns in
this data set are year, month, day, hour, location, gender, Res1, Res2.

    \begin{Verbatim}[commandchars=\\\{\}]
{\color{incolor}In [{\color{incolor}207}]:} \PY{c+c1}{\PYZsh{} load the tidyverse suite of packages }
          \PY{n}{library}\PY{p}{(}\PY{n}{tidyverse}\PY{p}{)}
          
          \PY{c+c1}{\PYZsh{} read the data into your workspace}
          \PY{n}{ba\PYZus{}data} \PY{o}{\PYZlt{}}\PY{o}{\PYZhy{}} \PY{n}{read\PYZus{}csv}\PY{p}{(}\PY{l+s+s1}{\PYZsq{}}\PY{l+s+s1}{datasets/breath\PYZus{}alcohol\PYZus{}ames.csv}\PY{l+s+s1}{\PYZsq{}}\PY{p}{)}
          
          \PY{c+c1}{\PYZsh{} quickly inspect the data}
          \PY{n}{head}\PY{p}{(}\PY{n}{ba\PYZus{}data}\PY{p}{)}
          
          \PY{c+c1}{\PYZsh{} obtain counts for each year }
          \PY{n}{ba\PYZus{}year} \PY{o}{\PYZlt{}}\PY{o}{\PYZhy{}} \PY{n}{ba\PYZus{}data} \PY{o}{\PYZpc{}}\PY{o}{\PYZgt{}}\PY{o}{\PYZpc{}} \PY{n}{count}\PY{p}{(}\PY{n}{year}\PY{p}{)}
          \PY{c+c1}{\PYZsh{}head(ba\PYZus{}year)}
\end{Verbatim}


    \begin{Verbatim}[commandchars=\\\{\}]
Parsed with column specification:
cols(
  year = col\_integer(),
  month = col\_integer(),
  day = col\_integer(),
  hour = col\_integer(),
  location = col\_character(),
  gender = col\_character(),
  Res1 = col\_double(),
  Res2 = col\_double()
)

    \end{Verbatim}

    \begin{tabular}{r|llllllll}
 year & month & day & hour & location & gender & Res1 & Res2\\
\hline
	 2017    & 12      & 17      & 1       & Ames PD & M       & 0.046   & 0.046  \\
	 2017    & 12      & 14      & 3       & ISU PD  & F       & 0.121   & 0.120  \\
	 2017    & 12      & 10      & 5       & ISU PD  & F       & 0.068   & 0.067  \\
	 2017    & 12      & 10      & 3       & ISU PD  & F       & 0.077   & 0.077  \\
	 2017    & 12      &  9      & 2       & ISU PD  & M       & 0.085   & 0.084  \\
	 2017    & 12      &  9      & 1       & Ames PD & M       & 0.160   & 0.161  \\
\end{tabular}


    
    \hypertarget{what-is-the-busiest-police-department-in-ames}{%
\subsection{2. What is the busiest police department in
Ames?}\label{what-is-the-busiest-police-department-in-ames}}

There are two police departments in the data set: the Iowa State
University Police Department and the Ames Police Department. Which one
administers more breathalyzer tests?

    \begin{Verbatim}[commandchars=\\\{\}]
{\color{incolor}In [{\color{incolor}209}]:} \PY{c+c1}{\PYZsh{} use count to tally up the totals for each department}
          \PY{n}{pds} \PY{o}{\PYZlt{}}\PY{o}{\PYZhy{}} \PY{n}{ba\PYZus{}data} \PY{o}{\PYZpc{}}\PY{o}{\PYZgt{}}\PY{o}{\PYZpc{}} \PY{n}{group\PYZus{}by}\PY{p}{(}\PY{n}{location}\PY{p}{)} \PY{o}{\PYZpc{}}\PY{o}{\PYZgt{}}\PY{o}{\PYZpc{}} 
                  \PY{n}{summarise}\PY{p}{(}\PY{n}{n} \PY{o}{=} \PY{n}{n}\PY{p}{(}\PY{p}{)}\PY{p}{)}
          \PY{n}{head}\PY{p}{(}\PY{n}{pds}\PY{p}{)}
\end{Verbatim}


    \begin{tabular}{r|ll}
 location & n\\
\hline
	 Ames PD & 616    \\
	 ISU PD  & 940    \\
\end{tabular}


    
    \hypertarget{nothing-good-happens-after-2am}{%
\subsection{3. Nothing Good Happens after
2am}\label{nothing-good-happens-after-2am}}

We all know that ``nothing good happens after 2am.'' Thus, there are
inevitably some times of the day when breath alcohol tests, especially
in a college town like Ames, are most and least common. Which hours of
the day have the most and least breathalyzer tests?

    \begin{Verbatim}[commandchars=\\\{\}]
{\color{incolor}In [{\color{incolor}211}]:} \PY{c+c1}{\PYZsh{} count by hour and arrange by descending frequency}
          \PY{n}{hourly} \PY{o}{\PYZlt{}}\PY{o}{\PYZhy{}} \PY{n}{ba\PYZus{}data} \PY{o}{\PYZpc{}}\PY{o}{\PYZgt{}}\PY{o}{\PYZpc{}} \PY{n}{group\PYZus{}by}\PY{p}{(}\PY{n}{hour}\PY{p}{)} \PY{o}{\PYZpc{}}\PY{o}{\PYZgt{}}\PY{o}{\PYZpc{}}
                      \PY{n}{summarise}\PY{p}{(}\PY{n}{n} \PY{o}{=} \PY{n}{n}\PY{p}{(}\PY{p}{)}\PY{p}{)} \PY{o}{\PYZpc{}}\PY{o}{\PYZgt{}}\PY{o}{\PYZpc{}}
                      \PY{n}{arrange}\PY{p}{(}\PY{n}{desc}\PY{p}{(}\PY{n}{n}\PY{p}{)}\PY{p}{)}
          \PY{c+c1}{\PYZsh{}head(hourly)}
          \PY{c+c1}{\PYZsh{} use a geom\PYZus{} to create the appropriate bar chart}
          \PY{n}{ggplot}\PY{p}{(}\PY{n}{hourly}\PY{p}{,} \PY{n}{aes}\PY{p}{(}\PY{n}{hour}\PY{p}{,} \PY{n}{n}\PY{p}{)}\PY{p}{)} \PY{o}{+} \PY{n}{geom\PYZus{}bar}\PY{p}{(}\PY{n}{stat} \PY{o}{=} \PY{l+s+s1}{\PYZsq{}}\PY{l+s+s1}{identity}\PY{l+s+s1}{\PYZsq{}}\PY{p}{)}
\end{Verbatim}


    
    
    \begin{center}
    \adjustimage{max size={0.9\linewidth}{0.9\paperheight}}{output_5_1.png}
    \end{center}
    { \hspace*{\fill} \\}
    
    \hypertarget{breathalyzer-tests-by-month}{%
\subsection{4. Breathalyzer tests by
month}\label{breathalyzer-tests-by-month}}

Now that we have discovered which time of day is most common for breath
alcohol tests, we will determine which time of the year has the most
breathalyzer tests. Which month will have the most recorded tests?

    \begin{Verbatim}[commandchars=\\\{\}]
{\color{incolor}In [{\color{incolor}213}]:} \PY{c+c1}{\PYZsh{} count by month and arrange by descending frequency}
          \PY{n}{monthly} \PY{o}{\PYZlt{}}\PY{o}{\PYZhy{}} \PY{n}{ba\PYZus{}data} \PY{o}{\PYZpc{}}\PY{o}{\PYZgt{}}\PY{o}{\PYZpc{}} \PY{n}{group\PYZus{}by}\PY{p}{(}\PY{n}{month}\PY{p}{)} \PY{o}{\PYZpc{}}\PY{o}{\PYZgt{}}\PY{o}{\PYZpc{}}
                      \PY{n}{summarise}\PY{p}{(}\PY{n}{n} \PY{o}{=} \PY{n}{n}\PY{p}{(}\PY{p}{)}\PY{p}{)} \PY{o}{\PYZpc{}}\PY{o}{\PYZgt{}}\PY{o}{\PYZpc{}}
                      \PY{n}{arrange}\PY{p}{(}\PY{n}{desc}\PY{p}{(}\PY{n}{n}\PY{p}{)}\PY{p}{)}
          \PY{c+c1}{\PYZsh{}head(monthly)}
          \PY{c+c1}{\PYZsh{} make month a factor}
          \PY{n}{monthly}\PY{err}{\PYZdl{}}\PY{n}{month} \PY{o}{\PYZlt{}}\PY{o}{\PYZhy{}} \PY{k}{as}\PY{o}{.}\PY{n}{factor}\PY{p}{(}\PY{n}{monthly}\PY{err}{\PYZdl{}}\PY{n}{month}\PY{p}{)}
          
          \PY{c+c1}{\PYZsh{} use a geom\PYZus{} to create the appropriate bar chart}
          \PY{n}{ggplot}\PY{p}{(}\PY{n}{monthly}\PY{p}{,} \PY{n}{aes}\PY{p}{(}\PY{n}{x} \PY{o}{=} \PY{n}{month}\PY{p}{,} \PY{n}{y} \PY{o}{=} \PY{n}{n}\PY{p}{)}\PY{p}{)} \PY{o}{+} \PY{n}{geom\PYZus{}bar}\PY{p}{(}\PY{n}{stat} \PY{o}{=} \PY{l+s+s1}{\PYZsq{}}\PY{l+s+s1}{identity}\PY{l+s+s1}{\PYZsq{}}\PY{p}{)}
\end{Verbatim}


    
    
    \begin{center}
    \adjustimage{max size={0.9\linewidth}{0.9\paperheight}}{output_7_1.png}
    \end{center}
    { \hspace*{\fill} \\}
    
    \hypertarget{college}{%
\subsection{5. COLLEGE}\label{college}}

When we think of (binge) drinking in college towns in America, we
usually think of something like this image at the left. And so, one
might suspect that breath alcohol tests are given to men more often than
women and that men drink more than women.

    \begin{Verbatim}[commandchars=\\\{\}]
{\color{incolor}In [{\color{incolor}215}]:} \PY{c+c1}{\PYZsh{} count by gender }
          \PY{n}{by\PYZus{}gender} \PY{o}{\PYZlt{}}\PY{o}{\PYZhy{}} \PY{n}{ba\PYZus{}data} \PY{o}{\PYZpc{}}\PY{o}{\PYZgt{}}\PY{o}{\PYZpc{}} \PY{n}{group\PYZus{}by}\PY{p}{(}\PY{n}{gender}\PY{p}{)} \PY{o}{\PYZpc{}}\PY{o}{\PYZgt{}}\PY{o}{\PYZpc{}}
                          \PY{n}{summarise}\PY{p}{(}\PY{n}{n} \PY{o}{=} \PY{n}{n}\PY{p}{(}\PY{p}{)}\PY{p}{)} \PY{o}{\PYZpc{}}\PY{o}{\PYZgt{}}\PY{o}{\PYZpc{}}
                          \PY{n}{arrange}\PY{p}{(}\PY{n}{desc}\PY{p}{(}\PY{n}{n}\PY{p}{)}\PY{p}{)}
          
          \PY{c+c1}{\PYZsh{} create a dataset with no NAs in gender }
          \PY{n}{clean\PYZus{}gender} \PY{o}{\PYZlt{}}\PY{o}{\PYZhy{}} \PY{n}{ba\PYZus{}data} \PY{o}{\PYZpc{}}\PY{o}{\PYZgt{}}\PY{o}{\PYZpc{}} \PY{n+nb}{filter}\PY{p}{(}\PY{n}{gender} \PY{o}{!=} \PY{l+s+s1}{\PYZsq{}}\PY{l+s+s1}{NA}\PY{l+s+s1}{\PYZsq{}}\PY{p}{)}
          \PY{c+c1}{\PYZsh{}head(clean\PYZus{}gender)}
          \PY{c+c1}{\PYZsh{} create a mean test result variable and save as mean\PYZus{}bas}
          \PY{n}{mean\PYZus{}bas} \PY{o}{\PYZlt{}}\PY{o}{\PYZhy{}} \PY{n}{clean\PYZus{}gender} \PY{o}{\PYZpc{}}\PY{o}{\PYZgt{}}\PY{o}{\PYZpc{}} \PY{n}{mutate}\PY{p}{(}\PY{n}{meanRes} \PY{o}{=} \PY{p}{(}\PY{n}{Res1} \PY{o}{+} \PY{n}{Res2}\PY{p}{)} \PY{o}{/} \PY{l+m+mi}{2}\PY{p}{)}
          \PY{c+c1}{\PYZsh{}head(mean\PYZus{}bas)}
          \PY{c+c1}{\PYZsh{} create side\PYZhy{}by\PYZhy{}side boxplots to compare the mean blood alcohol levels of men and women}
          \PY{n}{ggplot}\PY{p}{(}\PY{n}{mean\PYZus{}bas}\PY{p}{,} \PY{n}{aes}\PY{p}{(}\PY{n}{x}\PY{o}{=} \PY{n}{gender}\PY{p}{,} \PY{n}{y} \PY{o}{=} \PY{n}{meanRes}\PY{p}{)}\PY{p}{)} \PY{o}{+} \PY{n}{geom\PYZus{}boxplot}\PY{p}{(}\PY{p}{)}
\end{Verbatim}


    
    
    \begin{center}
    \adjustimage{max size={0.9\linewidth}{0.9\paperheight}}{output_9_1.png}
    \end{center}
    { \hspace*{\fill} \\}
    
    \hypertarget{above-the-legal-limit}{%
\subsection{6. Above the legal limit}\label{above-the-legal-limit}}

In the USA, it is illegal to drive with a blood alcohol concentration
(BAC) above 0.08\%. This is the case for all 50 states. Assuming
everyone tested in our data was driving (though we have no way of
knowing this from the data), if either of the results (Res1, Res2) are
above 0.08, the person would be charged with DUI (driving under the
influence).

    \begin{Verbatim}[commandchars=\\\{\}]
{\color{incolor}In [{\color{incolor}217}]:} \PY{c+c1}{\PYZsh{} Filter the data}
          \PY{n}{duis} \PY{o}{\PYZlt{}}\PY{o}{\PYZhy{}} \PY{n}{ba\PYZus{}data} \PY{o}{\PYZpc{}}\PY{o}{\PYZgt{}}\PY{o}{\PYZpc{}} \PY{n+nb}{filter}\PY{p}{(}\PY{n}{Res1} \PY{o}{\PYZgt{}} \PY{l+m+mf}{0.08} \PY{o}{|} \PY{n}{Res2} \PY{o}{\PYZgt{}} \PY{l+m+mf}{0.08} \PY{p}{)}
          
          \PY{c+c1}{\PYZsh{} proportion of tests that would have resulted in a DUI}
          \PY{n}{p\PYZus{}dui} \PY{o}{\PYZlt{}}\PY{o}{\PYZhy{}} \PY{n}{nrow}\PY{p}{(}\PY{n}{duis}\PY{p}{)} \PY{o}{/} \PY{n}{nrow}\PY{p}{(}\PY{n}{ba\PYZus{}data}\PY{p}{)}
          \PY{n}{p\PYZus{}dui}
\end{Verbatim}


    0.744858611825193

    
    \hypertarget{breathalyzer-tests-is-there-a-pattern-over-time}{%
\subsection{7. Breathalyzer tests: is there a pattern over
time?}\label{breathalyzer-tests-is-there-a-pattern-over-time}}

We previously saw that 2am is the most common time of day for
breathalyzer tests to be administered, and August is the most common
month of the year for breathalyzer tests. Now, we look at the weeks in
the year over time. We briefly use the lubridate package for a bit of
date-time manipulation.

    \begin{Verbatim}[commandchars=\\\{\}]
{\color{incolor}In [{\color{incolor}219}]:} \PY{n}{library}\PY{p}{(}\PY{n}{lubridate}\PY{p}{)} 
          
          \PY{c+c1}{\PYZsh{} Create date variable using paste() and ymd()}
          \PY{n}{ba\PYZus{}data} \PY{o}{\PYZlt{}}\PY{o}{\PYZhy{}} \PY{n}{ba\PYZus{}data} \PY{o}{\PYZpc{}}\PY{o}{\PYZgt{}}\PY{o}{\PYZpc{}} \PY{n}{mutate}\PY{p}{(}\PY{n}{date} \PY{o}{=} \PY{n}{ymd}\PY{p}{(}\PY{n}{paste}\PY{p}{(}\PY{n}{year}\PY{p}{,} \PY{n}{month}\PY{p}{,} \PY{n}{day}\PY{p}{)}\PY{p}{)}\PY{p}{)}
          
          \PY{c+c1}{\PYZsh{} Create a week variable using week()}
          \PY{n}{ba\PYZus{}data} \PY{o}{\PYZlt{}}\PY{o}{\PYZhy{}} \PY{n}{ba\PYZus{}data} \PY{o}{\PYZpc{}}\PY{o}{\PYZgt{}}\PY{o}{\PYZpc{}} \PY{n}{mutate}\PY{p}{(}\PY{n}{week} \PY{o}{=} \PY{n}{week}\PY{p}{(}\PY{n}{date}\PY{p}{)}\PY{p}{)}
          \PY{n}{head}\PY{p}{(}\PY{n}{ba\PYZus{}data}\PY{p}{)}
\end{Verbatim}


    \begin{tabular}{r|llllllllll}
 year & month & day & hour & location & gender & Res1 & Res2 & date & week\\
\hline
	 2017       & 12         & 17         & 1          & Ames PD    & M          & 0.046      & 0.046      & 2017-12-17 & 51        \\
	 2017       & 12         & 14         & 3          & ISU PD     & F          & 0.121      & 0.120      & 2017-12-14 & 50        \\
	 2017       & 12         & 10         & 5          & ISU PD     & F          & 0.068      & 0.067      & 2017-12-10 & 50        \\
	 2017       & 12         & 10         & 3          & ISU PD     & F          & 0.077      & 0.077      & 2017-12-10 & 50        \\
	 2017       & 12         &  9         & 2          & ISU PD     & M          & 0.085      & 0.084      & 2017-12-09 & 49        \\
	 2017       & 12         &  9         & 1          & Ames PD    & M          & 0.160      & 0.161      & 2017-12-09 & 49        \\
\end{tabular}


    
    \hypertarget{looking-at-timelines}{%
\subsection{8. Looking at timelines}\label{looking-at-timelines}}

How do the weeks differ over time? One of the most common data
visualizations is the time series, a line tracking the changes in a
variable over time. We will use the new week variable to look at test
frequency over time. We end with a time series plot showing frequency of
breathalyzer tests by week in year, with one line for each year.

    \begin{Verbatim}[commandchars=\\\{\}]
{\color{incolor}In [{\color{incolor}221}]:} \PY{c+c1}{\PYZsh{} create the weekly data set }
          \PY{n}{weekly} \PY{o}{\PYZlt{}}\PY{o}{\PYZhy{}} \PY{n}{ba\PYZus{}data} \PY{o}{\PYZpc{}}\PY{o}{\PYZgt{}}\PY{o}{\PYZpc{}} \PY{n}{group\PYZus{}by}\PY{p}{(}\PY{n}{week}\PY{p}{,} \PY{n}{year}\PY{p}{)} \PY{o}{\PYZpc{}}\PY{o}{\PYZgt{}}\PY{o}{\PYZpc{}}
                          \PY{n}{summarise}\PY{p}{(}\PY{n}{n} \PY{o}{=} \PY{n}{n}\PY{p}{(}\PY{p}{)}\PY{p}{)}
          
          
          \PY{c+c1}{\PYZsh{} uncomment and run the following line}
          \PY{n}{weekly} \PY{o}{\PYZlt{}}\PY{o}{\PYZhy{}} \PY{n}{weekly} \PY{o}{\PYZpc{}}\PY{o}{\PYZgt{}}\PY{o}{\PYZpc{}} \PY{n}{ungroup}\PY{p}{(}\PY{p}{)} \PY{c+c1}{\PYZsh{} ungroup is necessary for the plot later}
          \PY{c+c1}{\PYZsh{}head(weekly)}
          \PY{c+c1}{\PYZsh{} make year a factor}
          \PY{n}{weekly} \PY{o}{\PYZlt{}}\PY{o}{\PYZhy{}} \PY{n}{weekly} \PY{o}{\PYZpc{}}\PY{o}{\PYZgt{}}\PY{o}{\PYZpc{}} \PY{n}{mutate}\PY{p}{(}\PY{n}{year} \PY{o}{=} \PY{k}{as}\PY{o}{.}\PY{n}{factor}\PY{p}{(}\PY{n}{year}\PY{p}{)}\PY{p}{)}
          
          \PY{c+c1}{\PYZsh{} create the time series plot with one line for each year}
          \PY{n}{ggplot}\PY{p}{(}\PY{n}{weekly}\PY{p}{,} \PY{n}{aes}\PY{p}{(}\PY{n}{x} \PY{o}{=} \PY{n}{week}\PY{p}{,} \PY{n}{y} \PY{o}{=} \PY{n}{n}\PY{p}{)}\PY{p}{)} \PY{o}{+} 
            \PY{n}{geom\PYZus{}line}\PY{p}{(}\PY{p}{)} \PY{o}{+} 
            \PY{n}{geom\PYZus{}point}\PY{p}{(}\PY{n}{aes}\PY{p}{(}\PY{n}{color} \PY{o}{=} \PY{n}{year}\PY{p}{)}\PY{p}{)} \PY{o}{+}  \PY{c+c1}{\PYZsh{} included to make the plot more readable }
            \PY{n}{scale\PYZus{}x\PYZus{}continuous}\PY{p}{(}\PY{n}{breaks} \PY{o}{=} \PY{n}{seq}\PY{p}{(}\PY{l+m+mi}{0}\PY{p}{,}\PY{l+m+mi}{52}\PY{p}{,}\PY{l+m+mi}{2}\PY{p}{)}\PY{p}{)}  \PY{c+c1}{\PYZsh{} to make the x\PYZhy{}axis more readable }
\end{Verbatim}


    
    
    \begin{center}
    \adjustimage{max size={0.9\linewidth}{0.9\paperheight}}{output_15_1.png}
    \end{center}
    { \hspace*{\fill} \\}
    
    \hypertarget{the-end-of-veishea}{%
\subsection{9. The end of VEISHEA}\label{the-end-of-veishea}}

From Wikipedia: ``VEISHEA was an annual week-long celebration held each
spring on the campus of Iowa State University in Ames, Iowa. The
celebration featured an annual parade and many open-house demonstrations
of the university facilities and departments. Campus organizations
exhibited products, technologies, and held fundraisers for various
charity groups. In addition, VEISHEA brought speakers, lecturers, and
entertainers to Iowa State. {[}\ldots{}{]} VEISHEA was the largest
student-run festival in the nation, bringing in tens of thousands of
visitors to the campus each year.''

This over 90-year tradition in Ames was terminated permanently after
riots in 2014, where drunk celebrators flipped over multiple vehicles
and tore light poles down. This was not the first incidence of violence
and severe property damage in VEISHEA's history. Did former President
Leath make the right decision?

    \begin{Verbatim}[commandchars=\\\{\}]
{\color{incolor}In [{\color{incolor}223}]:} \PY{c+c1}{\PYZsh{}\PYZsh{} Run this code to create the plot }
          \PY{n}{ggplot}\PY{p}{(}\PY{p}{)} \PY{o}{+} 
            \PY{n}{geom\PYZus{}point}\PY{p}{(}\PY{n}{data} \PY{o}{=} \PY{n}{weekly}\PY{p}{,} \PY{n}{aes}\PY{p}{(}\PY{n}{x} \PY{o}{=} \PY{n}{week}\PY{p}{,} \PY{n}{y} \PY{o}{=} \PY{n}{n}\PY{p}{,} \PY{n}{color} \PY{o}{=} \PY{n}{year}\PY{p}{)}\PY{p}{)} \PY{o}{+} 
            \PY{n}{geom\PYZus{}line}\PY{p}{(}\PY{n}{data} \PY{o}{=} \PY{n}{weekly}\PY{p}{,} \PY{n}{aes}\PY{p}{(}\PY{n}{x} \PY{o}{=} \PY{n}{week}\PY{p}{,} \PY{n}{y} \PY{o}{=} \PY{n}{n}\PY{p}{,} \PY{n}{color} \PY{o}{=} \PY{n}{year}\PY{p}{)}\PY{p}{)} \PY{o}{+}  \PY{c+c1}{\PYZsh{} included to make the plot more readable }
            \PY{n}{geom\PYZus{}segment}\PY{p}{(}\PY{n}{data} \PY{o}{=} \PY{n}{NULL}\PY{p}{,} \PY{n}{arrow} \PY{o}{=} \PY{n}{arrow}\PY{p}{(}\PY{n}{angle} \PY{o}{=} \PY{l+m+mi}{20}\PY{p}{,} \PY{n}{length} \PY{o}{=} \PY{n}{unit}\PY{p}{(}\PY{l+m+mf}{0.1}\PY{p}{,} \PY{l+s+s2}{\PYZdq{}}\PY{l+s+s2}{inches}\PY{l+s+s2}{\PYZdq{}}\PY{p}{)}\PY{p}{,}
                                                    \PY{n}{ends} \PY{o}{=} \PY{l+s+s2}{\PYZdq{}}\PY{l+s+s2}{last}\PY{l+s+s2}{\PYZdq{}}\PY{p}{,} \PY{n+nb}{type} \PY{o}{=} \PY{l+s+s2}{\PYZdq{}}\PY{l+s+s2}{closed}\PY{l+s+s2}{\PYZdq{}}\PY{p}{)}\PY{p}{,} 
                         \PY{n}{aes}\PY{p}{(}\PY{n}{x} \PY{o}{=} \PY{n}{c}\PY{p}{(}\PY{l+m+mi}{20}\PY{p}{,}\PY{l+m+mi}{20}\PY{p}{)}\PY{p}{,} \PY{n}{xend} \PY{o}{=} \PY{n}{c}\PY{p}{(}\PY{l+m+mf}{15.5}\PY{p}{,}\PY{l+m+mi}{16}\PY{p}{)}\PY{p}{,} \PY{n}{y} \PY{o}{=} \PY{n}{c}\PY{p}{(}\PY{l+m+mi}{21}\PY{p}{,} \PY{l+m+mi}{20}\PY{p}{)}\PY{p}{,} \PY{n}{yend} \PY{o}{=} \PY{n}{c}\PY{p}{(}\PY{l+m+mi}{21}\PY{p}{,} \PY{l+m+mf}{12.25}\PY{p}{)}\PY{p}{)}\PY{p}{)} \PY{o}{+} 
            \PY{n}{geom\PYZus{}text}\PY{p}{(}\PY{n}{data} \PY{o}{=} \PY{n}{NULL}\PY{p}{,} \PY{n}{aes}\PY{p}{(}\PY{n}{x} \PY{o}{=} \PY{l+m+mi}{23}\PY{p}{,} \PY{n}{y} \PY{o}{=} \PY{l+m+mf}{20.5}\PY{p}{,} \PY{n}{label} \PY{o}{=} \PY{l+s+s2}{\PYZdq{}}\PY{l+s+s2}{VEISHEA Weeks}\PY{l+s+s2}{\PYZdq{}}\PY{p}{)}\PY{p}{,} \PY{n}{size} \PY{o}{=} \PY{l+m+mi}{3}\PY{p}{)} \PY{o}{+} 
            \PY{n}{scale\PYZus{}x\PYZus{}continuous}\PY{p}{(}\PY{n}{breaks} \PY{o}{=} \PY{n}{seq}\PY{p}{(}\PY{l+m+mi}{0}\PY{p}{,}\PY{l+m+mi}{52}\PY{p}{,}\PY{l+m+mi}{2}\PY{p}{)}\PY{p}{)} 
          
          \PY{c+c1}{\PYZsh{}\PYZsh{} Make a decision about VEISHEA. TRUE or FALSE?  }
          \PY{n}{cancelling\PYZus{}VEISHEA\PYZus{}was\PYZus{}right} \PY{o}{\PYZlt{}}\PY{o}{\PYZhy{}} \PY{n}{TRUE}
\end{Verbatim}


    
    
    \begin{center}
    \adjustimage{max size={0.9\linewidth}{0.9\paperheight}}{output_17_1.png}
    \end{center}
    { \hspace*{\fill} \\}
    

    % Add a bibliography block to the postdoc
    
    
    
    \end{document}
